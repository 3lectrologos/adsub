\section{Proofs}

\todo{Needs cleanup}

\subsection{Definitions}
\begin{description}[labelindent=0pt,leftmargin=7pt,itemindent=-2pt,itemsep=0pt]
  \uitem{Ground set} $E$
  \uitem{Observation set} $O$
  \uitem{Realization} $\phi : E \to O$
  \uitem{Partial realization} $\psi \subseteq E \times O$ with $\psi(e) \defeq o,\ \forall (e, o) \in \psi$
  \uitem{Partial realiz. domain} $\dom(\psi) \defeq \sdef{e \in E}{\exists o \in O : (e, o) \in \psi}$
  \uitem{Consistency} $\phi \sim \psi \iff \phi(e) = \psi(e),\ \forall e \in \dom(\psi)$
  \uitem{Subrealization} $\psi_1 \subseteq \psi_2$
  \uitem{Probability simplex} $\mathcal{P}(E) \defeq \sdef{\*x \in \mathbb{R}^E}{\sum_{e \in E} x_e = 1,\ x_e \geq 0\ \forall e \in E}$
  \uitem{Policy} $\pi \subseteq 2^{E \times O} \times \mathcal{P}(E)$ with $\pi(e\mid\psi) \defeq p(e),\ \forall (\psi, p) \in \pi$
  \uitem{Policy domain} $\dom(\pi) \defeq \sdef{\psi \in 2^{E \times O}}{\exists p \in \mathcal{P}(E) : (\psi, p) \in \pi}$
  \uitem{Truncated policy} $\pik \subseteq 2^{E \times O} \times \mathcal{P}(E)$ such that $\dom(\pik) = \sdef{\psi \in \dom(\pi)}{|\psi| < k}$ and $\pik(\psi) = \pi(\psi),\ \forall \psi \in \dom(\pik)$
  \uitem{Selected items} $E(\pi, \phi) \subseteq E$
  \uitem{Function} $f : 2^E \times O^E \to \mathbb{R}_{\geq 0}$
  \uitem{Exp. value of policy} $\favg(\pi) \defeq \E[\Phi,\Pi]{f(E(\pi, \Phi), \Phi)}$
  \uitem{Expected gain of element} $\D{e}{\psi} \defeq \E[\Phi]{f(\dom(\psi) \cup \{e\}, \Phi) - f(\dom(\psi), \Phi)}[\Phi \sim \psi]$
  \uitem{Expected gain of policy} $\D{\pi}{\psi} \defeq \E[\Phi,\Pi]{f(\dom(\psi) \cup E(\pi, \Phi), \Phi) - f(\dom(\psi), \Phi)}[\Phi \sim \psi]$
  \uitem{Policy concatenation} $\pi_1 @ \pi_2$ (note that $\favg(\pi_1 @ \pi_2) = \favg(\pi_2 @ \pi_1)$)
  \uitem{Random greedy policy} $\pig$
  \uitem{Random greedy set} $\mathcal{M}_k(\psi) \in \displaystyle\argmax_{S \subseteq E, |S| \leq k}\left\{\sum_{e \in S} \D{e}{\psi} \right\}$
\end{description}

\subsection{Problem statement}
We consider the problem of adaptive maximization, i.e., of finding a policy $\pi$ that solves the following problem
\begin{align*}
  \textrm{maximize}&\quad  \favg(\pi)\\
  \textrm{subject to}&\quad  |E(\pi, \phi)| \leq k,\ \forall \phi \in O^E.
\end{align*}

\section{Adaptive monotone case}

\begin{lemma}\label{lem:submod}
  If $f$ is adaptive submodular, then for any policy $\pi$ and any partial realization $\psi$ it holds that
  \begin{align*}
    \D{\pik}{\psi} \leq \sum_{e \in \mathcal{M}_k(\psi)} \D{e}{\psi}.
  \end{align*}
\end{lemma}
\begin{proof}
  The proof is similar to that of Lemma A.9 by~\citet{golovin11}, but presented here in more detail.
  First note that by definition of $\pik$ it holds that $|E(\pik, \phi)| \leq k,$ for all $\phi$, which also implies that $\E[\Phi,\Pi]{|E(\pik, \Phi)|} \leq k$.
  Let $p(e\mmid\psi)$ be the probability that element $e$ will be selected by the truncated policy $\pik$, which is run after having observed partial realization $\psi$, that is,
  \begin{align*}
    p(e\mmid\psi) \defeq \P[\Phi,\Pi]{e \in E(\pik, \Phi)}[\Phi \sim \psi].
  \end{align*}
  It follows that
  \begin{align}
    \label{eq:pre}
    \begin{split}
      k &\geq \E[\Phi, \Pi]{|E(\pik, \Phi)|}\\
        &= \sum_{e \in E}\P[\Phi,\Pi]{e \in E(\pik, \Phi)}\\
        &= \sum_{e \in E}p(e\mmid\psi).
    \end{split}
  \end{align}
  Now, consider the following fractional knapsack problem:
  \begin{align*}
    \textrm{maximize} & \quad g(\*w) \defeq \sum_{e \in E}\D{e}{\psi}w_e\\
    \textrm{subject to} & \quad \sum_{e \in E}w_e \leq k\\
               & \quad 0 \leq w_e \leq 1,\ \forall e \in E.
  \end{align*}
  Note that by \eqref{eq:pre} and the fact that $p(e\mmid\psi)$ are probabilities, it follows that $\*p \defeq \left(p(e\mmid\psi),\ e \in E\right)$ is a feasible vector for the above problem.
  Furthermore, the vector $\*m = (m_e)$ defined by
  \begin{align*}
    m_e = \twopartdefo{1}{e \in \mathcal{M}_k(\psi)}{0}
  \end{align*}
  is an optimal solution. \todo{Could add a proof by contradiction here if this isn't obvious.}
  Therefore, we have
  \begin{align}\label{eq:knap}
                    &\ g(\*p) \leq g(\*m) \notag\\
    \Leftrightarrow &\ \sum_{e \in E}\D{e}{\psi}w_e \leq \sum_{e \in \mathcal{M}_k(\psi)} \D{e}{\psi}
  \end{align}
  
  Let $p(\psi'\mmid\psi)$ be the probability that partial realization $\psi' \supseteq \psi$ will come up when running policy $\pik$ given partial realization $\psi$, that is,
  \begin{align*}
    p(\psi'\mmid\psi) \defeq \P[\Phi, \Pi]{\sdef{\left(e, \Phi(e)\right)}{e \in E(\pik, \Phi)} = \psi'\setminus\psi}[\Phi \sim \psi].
  \end{align*}
  Then, the gain $\D{\pik}{\psi}$ can be bounded as follows:
  \begin{align*}
    \D{\pik}{\psi} &= \sum_{\psi' \in \dom(\pik)}p(\psi'\mmid\psi)\sum_{e \in E}\pik(e\mmid\psi')\D{e}{\psi'}\\
    &\leq \sum_{\psi' \in \dom(\pik)}p(\psi'\mmid\psi)\sum_{e \in E}\pik(e\mmid\psi')\D{e}{\psi} \tag*{by AS}\\
    &= \sum_{e \in E}\D{e}{\psi} \sum_{\psi' \in \dom(\pik)}\pik(e\mmid\psi')p(\psi'\mmid\psi)\\
    &= \sum_{e \in E}\D{e}{\psi} p(e\mmid\psi)\\
    &\leq \sum_{e \in \mathcal{M}_k(\psi)} \D{e}{\psi} \tag*{by \eqref{eq:knap}}.
  \end{align*}
\end{proof}

\begin{lemma}\label{lem:mon_main}
  For any policy $\pi$ and any non-negative integer $i < k$, if $f$ is adaptive submodular, the expected marginal gain obtained at the $i$-th step of random greedy policy $\pig$ can bounded as
  \begin{align*}
    \favg(\pigii) - \favg(\pigi) \geq \frac{1}{k}\left(\favg(\pigi @ \pi) - \favg(\pigi)\right).
  \end{align*}
\end{lemma}
\begin{proof}
  Fix $i < k$ and let $\Psi$ be a random variable denoting the partial realization that results from running the random greedy policy for $i$ steps, distributed as
  \begin{align*}
    \P[\Psi]{\Psi = \psi} = \P[\Phi, \Pi]{\sdef{\left(e, \Phi(e)\right)}{e \in E(\pigi, \Phi)} = \psi}.
  \end{align*}
  Also, let $U_i$ be a random variable denoting the element chosen at the $i$-th step of the random greedy policy. Due to the way the random greedy policy selects the next element at each step, the distribution of $U_{i+1}$ conditioned on some partial realization $\psi$ up to step $i$ is
  \begin{align}\label{eq:ui}
    \P[\Pi]{U_{i+1} = e} = \twopartdefo{1/k}{e \in \mathcal{M}_k(\psi)}{0}.
  \end{align}
  
  Then, for the expected marginal gain at the $i$-th step we have
  \begin{align*}
     &\ \favg(\pigii) - \favg(\pigi)\\
    =&\ \E[\Phi, \Pi]{f(E(\pigii, \Phi), \Phi) - f(E(\pigi, \Phi), \Phi)}\\
    =&\ \E[\Psi, \Phi, \Pi]{f(\dom(\Psi) \cup \{U_{i+1}\}, \Phi) - f(\dom(\Psi), \Phi)}[\Phi \sim \Psi]\\
    =&\ \E[\Psi, \Phi]{\sum_{e \in \mathcal{M}_k(\Psi)}\frac{1}{k}\left[f(\dom(\Psi) \cup \{e\}, \Phi) - f(\dom(\Psi), \Phi)\right]}[\Phi \sim \Psi] \tag*{by \eqref{eq:ui}}\\
    =&\ \frac{1}{k}\E[\Psi]{\sum_{e \in \mathcal{M}_k(\Psi)}\D{e}{\Psi}}\\
    \geq&\ \frac{1}{k}\E[\Psi]{\D{\pi}{\Psi}} \tag*{by~\lemmaref{lem:submod}}\\
    =&\ \frac{1}{k}\E[\Psi, \Phi]{f(\dom(\Psi) \cup E(\pi, \Phi), \Phi) - f(\dom(\Psi), \Phi)}[\Phi \sim \Psi]\\
    =& \ \frac{1}{k}\left(\favg(\pigi @ \pi) - \favg(\pigi)\right).
  \end{align*}
\end{proof}

\begin{lemma}\label{lem:mon_mon}
  Function $f$ is adaptive monotone if and only if for all policies $\pi_1$ and $\pi_2$ it holds that
  \begin{align*}
    \favg(\pi_2) \leq \favg(\pi_1 @ \pi_2).
  \end{align*}
\end{lemma}
\begin{proof}
See Lemma A.8 of \citet{golovin11}.
\end{proof}

\begin{theorem}
  If $f$ is adaptive monotone submodular, then for any policy $\pi$ and all integers $i, k > 0$ it holds that
  \begin{align*}
    \favg(\pigi) \geq \left(1 - e^{-i/k}\right)\favg(\pik).
  \end{align*}
\end{theorem}
\begin{proof}
  By combining \lemmasref{lem:mon_main} and~\ref{lem:mon_mon} it immediately follows that for all $i$, $k \geq 1$
  \begin{align*}
                   &\ \favg(\pigii) - \favg(\pigi) \geq \frac{1}{k}\left(\favg(\pik) - \favg(\pigi)\right)\\
    \Leftrightarrow&\ \favg(\pigii) \geq \frac{1}{k}\favg(\pik) + \left(1 - \frac{1}{k}\right) \favg(\pigi)\\
    \Leftrightarrow&\ \favg(\pik) - \favg(\pigii) \leq \left(1 - \frac{1}{k}\right)\left(\favg(\pik) - \favg(\pigi)\right)\\
    \Leftrightarrow&\ \favg(\pik) - \favg(\pigi) \leq \left(1 - \frac{1}{k}\right)^{i}\left(\favg(\pik) - \favg(\pigo)\right)\\
    \Leftrightarrow&\ \favg(\pigi) \geq \left(1 - \left(1 - \frac{1}{k}\right)^{i}\right)\favg(\pik)\tag*{by non-negativity of $f$}\\
    \Leftrightarrow&\ \favg(\pigi) \geq \left(1 - e^{-i/k}\right)\favg(\pik)\tag*{($1 - x \leq e^{-x},\ \forall x \geq 0$)}
  \end{align*}
\end{proof}

\begin{cor}
  If $f$ is adaptive monotone submodular, then for any policy $\pi$ and any integer $k > 0$ it holds that
  \begin{align*}
    \favg(\pigk) \geq (1 - e^{-1})\favg(\pik).
  \end{align*}
\end{cor}

\section{Non-monotone case}
\begin{lemma}\label{lem:buch}
  If $f : 2^E \to \mathbb{R}_{\geq 0}$ is submodular and $A$ is a random subset of $E$, such that each element $e \in E$ is contained in $A$ with probability at most $p$, that is, $\P[A]{e \in A} \leq p,\ \forall e \in E$, then it holds that
  \begin{align*}
    \E[A]{f(A)} \geq (1-p)f(\varnothing).
  \end{align*}
\end{lemma}
\begin{proof}
  See Lemma 2.2 of \citet{buchbinder14}.
\end{proof}

\begin{lemma}\label{lem:nmon}
  If $f(\cdot\,, \phi) : 2^E \to \mathbb{R}_{\geq 0}$ is submodular for all $\phi \in O^E$, then for any policy $\pi$ such that each element of $e \in E$ is selected by it with probability at most $p$, that is, $\P[\Pi]{e \in E(\pi, \phi)} \leq p,\ \forall \phi \in O^E,\ \forall e \in E$, the expected value of running $\pi$ can be bounded as follows:
\begin{align*}
  \favg(\pi) \geq (1-p)\,\favg(\pio).
\end{align*}
\end{lemma}
\begin{proof}
  \begin{align*}
    \favg(\pi) &= \E[\Phi,\Pi]{f(E(\pi, \Phi), \Phi)}\\
               &= \E[\Phi]{\E[\Pi]{f(E(\pi, \Phi), \Phi)}}\\
               &\geq \E[\Phi]{(1-p)f(\varnothing, \Phi)} \tag*{by \lemmaref{lem:buch}}\\
               &= (1-p)\favg(\pio)
  \end{align*}
\end{proof}

\begin{cor}\label{cor:nmon}
  If $f(\cdot\,, \phi) : 2^E \to \mathbb{R}_{\geq 0}$ is submodular for all $\phi \in O^E$, then for any policy $\pi$ such that each element of $e \in E$ is selected by it with probability at most $p$, that is, $\P[\Pi]{e \in E(\pi, \phi)} \leq p,\ \forall \phi \in O^E,\ \forall e \in E$, and any policy $\pi'$, the expected value of running $\pi'@\pi$ can be bounded as follows:
\begin{align*}
  \favg(\pi' @ \pi) \geq (1-p)\favg(\pi').
\end{align*}
\end{cor}
\todo{Add proof?}

\begin{lemma}\label{lem:sel}
  For every $\phi \in O^E$, after running the random greedy policy for $i$ steps, the probability of any element $e \in E$ having been selected can be bounded  as follows:
\begin{align*}
  \P[\Pi]{e \in E(\pigi, \phi)} \leq 1 - \left(1 - \frac{1}{k}\right)^i
\end{align*}
\end{lemma}
\begin{proof}
  Similarly to Observation 3.2 by \citet{buchbinder14}, at each step of the random greedy policy, an element $e$ is \emph{not} selected by the policy with probability at least $1 - 1/k$. After $i$ steps we have
  \begin{align*}
    \P[\Pi]{e \not\in E(\pigi, \phi)} \geq \left(1 - \frac{1}{k}\right)^i \Rightarrow\ \P[\Pi]{e \in E(\pigi, \phi)} \leq 1 - \left(1 - \frac{1}{k}\right)^i
  \end{align*}
\end{proof}

\begin{theorem}\label{thm:nonm}
  If $f(\cdot\,, \phi) : 2^E \to \mathbb{R}_{\geq 0}$ is submodular for all $\phi \in O^E$, then for any policy $\pi$ and all integers $i, k > 0$ it holds that
  \begin{align*}
    \favg(\pigi) \geq \frac{i}{k}\left(1 - \frac{1}{k}\right)^{i-1}\favg(\pik).
  \end{align*}
\end{theorem}
\begin{proof}
  For any policy $\pi$, we can use \lemmaref{lem:sel} to apply \corref{cor:nmon} to bound the expected value of $\pigi @ \pi$ as follows:
  \begin{align}\label{eq:rec}
    \favg(\pigi @ \pi) = \favg(\pi @ \pigi) \geq \left(1 - \frac{1}{k}\right)^i \favg(\pi).
  \end{align}
  Then, \lemmaref{lem:mon_main} gives
  \begin{align*}
    &\favg(\pigii) - \favg(\pigi) \geq \frac{1}{k}\left(\favg(\pigi @ \pi) - \favg(\pigi)\right)\\
    \Rightarrow\ &\favg(\pigii) - \favg(\pigi) \geq \frac{1}{k}\left(1 - \frac{1}{k}\right)^{i-1}\favg(\pi) - \frac{1}{k}\favg(\pigi) \tag*{by \eqref{eq:rec}}
  \end{align*}
  From the last equation the theorem follows by induction as in the proof of Theorem 1.3 by \citet{buchbinder14}.
\end{proof}

\begin{cor}
  If $f(\cdot\,, \phi)$ is submodular for all $\phi \in O^E$, then for any policy $\pi$ and any integer $k > 0$ it holds that
  \begin{align*}
    \favg(\pigk) \geq e^{-1}\favg(\pik).
  \end{align*}
\end{cor}
