%%%% ijcai15.tex

\typeout{Non-monotone Adaptive Submodular Maximization}

% These are the instructions for authors for IJCAI-15.
% They are the same as the ones for IJCAI-11 with superficical wording
%   changes only.

\documentclass{article}
% The file ijcai15.sty is the style file for IJCAI-15 (same as ijcai07.sty).
\usepackage{ijcai15}

% Use the postscript times font!
\usepackage{times}

% the following package is optional:
%\usepackage{latexsym} 

% Following comment is from ijcai97-submit.tex:
% The preparation of these files was supported by Schlumberger Palo Alto
% Research, AT\&T Bell Laboratories, and Morgan Kaufmann Publishers.
% Shirley Jowell, of Morgan Kaufmann Publishers, and Peter F.
% Patel-Schneider, of AT\&T Bell Laboratories collaborated on their
% preparation.

% These instructions can be modified and used in other conferences as long
% as credit to the authors and supporting agencies is retained, this notice
% is not changed, and further modification or reuse is not restricted.
% Neither Shirley Jowell nor Peter F. Patel-Schneider can be listed as
% contacts for providing assistance without their prior permission.

% To use for other conferences, change references to files and the
% conference appropriate and use other authors, contacts, publishers, and
% organizations.
% Also change the deadline and address for returning papers and the length and
% page charge instructions.
% Put where the files are available in the appropriate places.

%-------------------------------------------------------------------------------
% Custom definitions
%-------------------------------------------------------------------------------
\usepackage{etoolbox}
% Define output mode (default: six-page IJCAI)
\newtoggle{short}
\toggletrue{short}
% Uncomment for long version
\togglefalse{short}

\usepackage[usenames,dvipsnames]{xcolor}
\usepackage{amsmath,amssymb,amsthm}
\usepackage{mathtools}                    % For \vcentcolon
\usepackage{xspace}            % Controls space after user-defined command
\usepackage{xparse}            % Optional arguments in commands
\usepackage{enumitem}
\usepackage{fixltx2e}          % For subscripts in normal text
\usepackage{bm}
\usepackage{algorithm}
\usepackage{algorithmic}
\usepackage{pgfplots}
\iftoggle{short}
{
\usepackage{nohyperref}
\usepackage{url}
}
{
\usepackage{flushend}
\usepackage[debug]{hyperref}
\definecolor{mydarkblue}{rgb}{0,0.08,0.45}
\hypersetup{%
  pdftitle={Active Learning for Level Set Estimation},
  pdfauthor={},
  pdfsubject={},
  pdfkeywords={},
  pdfborder=0 0 0,
  pdfpagemode=UseNone,
  colorlinks=true,
  linkcolor=mydarkblue,
  citecolor=mydarkblue,
  filecolor=mydarkblue,
  urlcolor=mydarkblue,
  pdfview=FitH}
}

\newcommand{\todo}[1]{\noindent\texttt{\small\color[rgb]{0.5,0.1,0.1} TODO: #1}}

% Refs
\newcommand{\sectref}[1]{\hyperref[#1]{Section \ref*{#1}}}
\newcommand{\chapref}[1]{\hyperref[#1]{Chapter \ref*{#1}}}
\newcommand{\figref}[1]{\hyperref[#1]{Figure \ref*{#1}}}
\newcommand{\figsref}[1]{\hyperref[#1]{Figures \ref*{#1}}}
\newcommand{\tabref}[1]{\hyperref[#1]{ Table \ref*{#1}}}
\newcommand{\algoref}[1]{\hyperref[#1]{Algorithm \ref*{#1}}}
\newcommand{\theoremref}[1]{\hyperref[#1]{Theorem \ref*{#1}}}
\newcommand{\lemmaref}[1]{\hyperref[#1]{Lemma \ref*{#1}}}
\newcommand{\lemmasref}[1]{\hyperref[#1]{Lemmas \ref*{#1}}}
\newcommand{\corref}[1]{\hyperref[#1]{Corollary \ref*{#1}}}
\newcommand{\asref}[1]{\hyperref[#1]{Assumption \ref*{#1}}}
\newcommand{\eqtref}[1]{\hyperref[#1]{\mbox{(\ref*{#1})}}}
\newcommand{\appref}[1]{\hyperref[#1]{Appendix \ref*{#1}}}
\newcommand{\lineref}[1]{\hyperref[#1]{line \ref*{#1}}}
\newcommand{\linesref}[2]{\hyperref[#1]{lines \ref*{#1}--\ref*{#2}}}
\newcommand{\linsref}[1]{\hyperref[#1]{lines \ref*{#1}}}

% Theory environments
\newtheorem{definition}{Definition}
\newtheorem{theorem}{Theorem}
\newtheorem{lemma}{Lemma}
\newtheorem{cor}{Corollary}
\newtheorem{assumption}{Assumption}

% Algorithm-related
\newcommand{\theHalgorithm}{\arabic{algorithm}}
\renewcommand{\algorithmicrequire}{\textbf{Input:}}
\renewcommand{\algorithmicensure}{\textbf{Output:}}
\newcommand*\LNot{\textbf{not}\xspace}
\newcommand*\LAnd{\textbf{and}\xspace}
\newcommand*\LET[2]{\STATE #1 $\gets$ #2}
\newcommand*\Fcall[1]{\textsc{#1}}
\makeatletter
\newcommand{\setalglineno}[1]{%
  \setcounter{ALC@line}{\numexpr#1-1}}
\makeatother
\renewcommand{\algorithmiccomment}[1]{// #1}
\newcommand{\LINEIF}[2]{%
    \STATE\algorithmicif\ {#1}\ \algorithmicthen\ {#2}%
}
\newcommand{\LINEELSE}[1]{%
    \STATE\algorithmicelse\ {#1}%
}

% Math definitions
\newcommand{\argmax}{\operatornamewithlimits{argmax}}
\newcommand{\argmin}{\operatornamewithlimits{argmin}}
\def\*#1{\bm{#1}}

\newcommand{\twopartdef}[4]
{
	\left\{
		\begin{array}{ll}
			#1\,,& \mbox{if } #2 \\
			#3\,,& \mbox{if } #4
		\end{array}
	\right.
}

\newcommand{\twopartdefo}[3]
{
	\left\{
		\begin{array}{ll}
			#1\,,& \mbox{if } #2 \\
			#3\,,& \mbox{otherwise}
		\end{array}
	\right.
}

\newcommand{\favg}{f_{\mathrm{avg}}}
\DeclareMathOperator{\dom}{dom}
\newcommand{\defeq}{\vcentcolon=}

% Fix spacing problem with \left and \right
\let\originalleft\left
\let\originalright\right
\renewcommand{\left}{\mathopen{}\mathclose\bgroup\originalleft}
\renewcommand{\right}{\aftergroup\egroup\originalright}

% Math definitions
\newcommand{\smid}{\ \middle\vert\ }
\newcommand{\mmid}{\,\vert\,}

\DeclareDocumentCommand \E { o m o }{%
  \IfValueTF {#1} {%
    \mathbb{E}_{#1}
  }{%
    \mathbb{E}
  }%
  \left[#2
  \IfValueT {#3} {%
    \smid #3
  }%
  \right]
}

\DeclareDocumentCommand \P { o m o }{%
  \IfValueTF {#1} {%
    \mathbb{P}_{#1}
  }{%
    \mathbb{P}
  }%
  \left[#2
  \IfValueT {#3} {%
    \smid #3
  }%
  \right]
}

\newcommand{\D}[2]{\Delta(#1\mmid#2)}
\newcommand{\sdef}[2]{\left\{#1\smid#2\right\}}

\newcommand{\pio}{\pi_{[0]}}
\newcommand{\pii}{\pi_{[i]}}
\newcommand{\pik}{\pi_{[k]}}
\newcommand{\pis}{\pi^*}
\newcommand{\pisi}{\pi^*_{[i]}}
\newcommand{\pisk}{\pi^*_{[k]}}
\newcommand{\pig}{\pi^{\textrm{rg}}}
\newcommand{\pigo}{\pi^{\textrm{rg}}_{[0]}}
\newcommand{\pigi}{\pi^{\textrm{rg}}_{[i]}}
\newcommand{\pigl}{\pi^{\textrm{rg}}_{[\ell]}}
\newcommand{\pigii}{\pi^{\textrm{rg}}_{[i+1]}}
\newcommand{\pigk}{\pi^{\textrm{rg}}_{[k]}}

\newcommand{\uitem}[1]{\item[#1]}

% For citations like: Author et. al [2010]
\newcommand{\citet}[1]{\citeauthor{#1}~\shortcite{#1}}

% Paragraph
\renewcommand{\paragraph}[1]{\vspace{0.3em}\noindent\textbf{#1.}\makebox[0.5em]{}}
%-------------------------------------------------------------------------------

\title{Non-monotone Adaptive Submodular Maximization}
\author{Alkis Gotovos\\
ETH Zurich
\And
Amin Karbasi\\
Yale University
\And
Andreas Krause\\
ETH Zurich}

\begin{document}

\maketitle

\begin{abstract}
\todo{abstract}
\end{abstract}

\section{Introduction}

\subsection{Related work}
Test \cite{golovin11} and \citet{buchbinder14}

\section{Problem Statement}
We now formally introduce the class of problems of interest.
Assume we are given a finite ground set $E$ and a set $O$ of observable states.
Each item $e \in E$ is associated with a state $o \in O$ through a function $\phi : E \to O$, which is called a realization of the ground set.
In our setting, we assume that the realization $\Phi$ is a random variable with known distribution $p(\phi)$.
Furthermore, we are given an objective function $f : 2^E \times O^E \to \mathbb{R}_{\geq 0}$.
For a set $A \subseteq E$ and a realization $\phi$, the quantity $f(A, \phi)$ expresses the utility we get by selecting subset $A$, if the true realization is $\phi$.

Our goal is to come up with a sequential policy that builds up a set $A \subseteq E$, such that our utility $f(A, \Phi)$ is maximized.
That is, we iteratively select an item $e \in E$ to add to $A$ and observe its state $\Phi(e)$.
In this setting, there are two factors that complicate matters compared to its non-adaptive counterpart.
First, since our utility depends on the (random) realization, we need to maximize the expected utility under the distribution of realizations $p(\phi)$.
Second, the chosen set $A$ itself is a random variable that depends on the realization, since the choices of our policy will change according to each observation $\Phi(e)$, which is, of course, the whole point of adaptivity.
In addition, the policy itself might make random decisions, which is an additional source of randomness for $A$.

To address the above, we define a partial realization as a set $ \psi \subseteq E \times O$, which represents the item-observation pairs over a subset of $E$.
In particular, we call this subset the \emph{domain} of $\psi$, which is formally defined as $\dom(\psi) \defeq \sdef{e \in E}{\exists o \in O : (e, o) \in \psi}$.
Additionally, we write $\psi(e) = o$, if $(e, o) \in \psi$ and call $\psi$ \emph{consistent} with realization $\phi$, if $\psi(e) = \phi(e)$, for all $e \in \dom(\psi)$, which means that the observations of a subset according to $\psi$ agree with the assignments over the whole ground set according to $\phi$.

Now, we can define a \emph{policy} $\pi$ as a function from partial realizations to a distribution over which item to pick next, formally, $\pi : 2^{E \times O} \to \mathcal{P}(E)$.
The policy terminates when the current partial realization is not in its domain denoted by $\dom(\pi) \subseteq 2^{E \times O}$.
We use the notation $\pi(e\mmid\psi)$ for the probability of picking item $e$ given partial realization $\psi$.
We call $E(\pi, \Phi) \subseteq E$ the set of items that have been selected upon termination of policy $\pi$ under realization $\Phi$.
Note that this set a random variable that depends on both the randomness of the policy, as well as the randomness of the realization.

Finally, we can formalize the performance of a policy $\pi$ by its expected utility $\favg(\pi) \defeq \E[\Phi,\Pi]{f(E(\pi, \Phi), \Phi)}$.
Then, our goal is to come up with a policy that maximizes the expected utility, subject to a cardinality constraint on the number of items to be picked, $|E(\pi, \Phi)| \leq k$.

\section{The Adaptive Random Greedy Algorithm}
\todo{Need to mention $2k$ dummy elements added to $E$.}

\begin{algorithm}[!b]
  \caption{Adaptive random greedy}
  \label{alg:rg}
\normalsize{
\begin{algorithmic}[1]
  \REQUIRE ground set $E$, function $f$, distribution $p(\phi)$, cardinality constraint $k$
%  \ENSURE set $S_k \subseteq E$
  \LET{$A$}{$\varnothing$}
  \LET{$\psi$}{$\varnothing$}
  \FOR{$i = 1$ \TO $k$}
    \STATE Compute $\D{e}{\psi}$, for all $e \in E$ \label{lin:marg}
    \LET{$\mathcal{M}_k(\psi)$}{$\displaystyle\argmax_{S \subseteq E, |S| \leq k}\left\{\sum_{e \in S} \D{e}{\psi} \right\}$} \label{lin:argmax}
    \STATE Sample $m$ uniformly at random from $\mathcal{M}_k(\psi)$
    \LET{$A$}{$A \cup \{m\}$}
    \STATE Observe $\Phi(m)$
    \LET{$\psi$}{$\psi \cup \left\{\big(m, \Phi(m)\big)\right\}$}
  \ENDFOR
  \STATE Return $A$
\end{algorithmic}
}
\end{algorithm}

\section{Theoretical Analysis}

\section{Experiments}

\section{Conclusion}

% The file named.bst is a bibliography style file for BibTeX 0.99c
\bibliographystyle{named}
\bibliography{ijcai15}

\iftoggle{short}
{}
{
\clearpage
\onecolumn
\appendix
\appendix
\chapter{Proofs}

\section{Definitions}

\begin{itemize}
  \item[\underline{Ground set}] $E$
  \item[\underline{Observation set}] $O$
  \item[\underline{Realization}] $\phi : E \to O$
  \item[\underline{Partial realization}] $\psi \subseteq E \times O$ with $\psi(e) \defeq o,\ \forall (e, o) \in \psi$
  \item[\underline{Partial realiz. domain}] $\dom(\psi) \defeq \sdef{e \in E}{\exists o \in O : (e, o) \in \psi}$
  \item[\underline{Consistency}] $\phi \sim \psi \iff \phi(e) = \psi(e),\ \forall e \in \dom(\psi)$
  \item[\underline{Subrealization}] $\psi_1 \subseteq \psi_2$
  \item[\underline{Probability simplex}] $\mathcal{P}(E) \defeq \sdef{\*x \in \mathbb{R}^E}{\sum_{e \in E} x_e = 1,\ x_e \geq 0\ \forall e \in E}$
  \item[\underline{Policy}] $\pi \subseteq 2^{E \times O} \times \mathcal{P}(E)$ with $\pi(e\mid\psi) \defeq p(e),\ \forall (\psi, p) \in \pi$
  \item[\underline{Policy domain}] $\dom(\pi) \defeq \sdef{\psi \in 2^{E \times O}}{\exists p \in \mathcal{P}(E) : (\psi, p) \in \pi}$
  \item[\underline{Truncated policy}] $\pik \subseteq 2^{E \times O} \times \mathcal{P}(E)$ such that $\dom(\pik) = \sdef{\psi \in \dom(\pi)}{|\psi| < k}$ and $\pik(\psi) = \pi(\psi),\ \forall \psi \in \dom(\pik)$
  \item[\underline{Selected items}] $E(\pi, \phi) \subseteq E$
  \item[\underline{Function}] $f : 2^E \times O^E \to \mathbb{R}_{\geq 0}$
  \item[\underline{Exp. value of policy}] $\favg(\pi) \defeq \E[\Phi,\Pi][f(E(\pi, \Phi), \Phi)]$
  \item[\underline{Marginal gain of $e$}] $\D{e}{\psi} \defeq \E[\Phi]{f(\dom(\psi) \cup \{e\}, \Phi) - f(\dom(\psi), \Phi)}[\Phi \sim \psi]$
  \item[\underline{Marginal gain of $\pi$}] $\D{\pi}{\psi} \defeq \E[\Phi,\Pi]{f(\dom(\psi) \cup E(\pi, \Phi), \Phi) - f(\dom(\psi), \Phi)}[\Phi \sim \psi]$
  \item[\underline{Random greedy set}] $\mathcal{M}_k(\psi) \in \displaystyle\argmax_{S \subseteq E, |S| = k}\left\{\sum_{e \in S} \D{e}{\psi} \right\}$
\end{itemize}


\section{Monotone adaptive submodular}

\todo{Need to add dummy elements, so that it's always possible to choose $k$ of them with non-negative expected marginal gains.}

\begin{lemma}
  For any partial realization $\psi$, let $\mathcal{M}_k(\psi)$ be a set containing the $k$ elements with the largest expected marginal gains.
  If $f$ is adaptive submodular, then for any policy $\pi$ it holds that
  \begin{align*}
    \D{\pik}{\psi} \leq \sum_{e \in \mathcal{M}_k(\psi)} \D{e}{\psi}.
  \end{align*}
\end{lemma}
\begin{proof}
  First note that by definition of $\pik$ it holds that $|E(\pik, \phi)| \leq k,$ for all $\phi$, which also implies that $\E[\Phi,\Pi]{|E(\pik, \Phi)|} \leq k$.
  Let $p(e)$ denote the probability that element $e$ will be selected by the truncated policy $\pik$, that is, $p(e) \defeq \P[\Phi,\Pi]{e \in E(\pik, \Phi)}$.
  Then, we have
  \begin{align}
    \label{eq:pre}
    \begin{split}
      k &\geq \E[\Phi]{|E(\pik, \Phi)|}\\
        &= \sum_{e \in E}\P[\Phi,\Pi]{e \in E(\pik, \Phi)}\\
        &= \sum_{e \in E}p(e).
    \end{split}
  \end{align}
  If we define $p(\psi'\mmid\psi) \defeq \P[\Phi,\Pi]{\dom(\psi') \subseteq E(\pik, \Phi)}[\Phi \sim \psi]$, i.e., the probability that partial realization $\psi' \supseteq \psi$ will come up when running policy $\pik$, the gain $\D{\pik}{\psi}$ can be bounded as follows:
  \begin{align*}
    \D{\pik}{\psi} &= \sum_{\psi' \in \dom(\pik)}p(\psi'\mmid\psi)\sum_{e \in E}\pi(e\mmid\psi')\D{e}{\psi'}\\
    &\leq \sum_{\psi' \in \dom(\pik)}p(\psi'\mmid\psi)\sum_{e \in E}\pi(e\mmid\psi')\D{e}{\psi} \tag*{(by AS)}\\
    &= \sum_{e \in E}\D{e}{\psi} \sum_{\psi' \in \dom(\pik)}\pi(e\mmid\psi')p(\psi'\mmid\psi)\\
    &= \sum_{e \in E}\D{e}{\psi} p(e)\\
    &\leq \sum_{e \in \mathcal{M}_k(\psi)} \D{e}{\psi}
  \end{align*}
  The last inequality above follows from considering the fractional knapsack problem
  \begin{align*}
    \textrm{maximize} & \quad \sum_{e \in E}p(e)\D{e}{\psi}\\
    \textrm{subject to} & \quad \sum_{e \in E}p(e) \leq k \tag*{(by \eqref{eq:pre})}\\
               & \quad 0 \leq p(e) \leq 1,\ \forall e \in E \tag*{($p(e)$ are probabilities)}
  \end{align*}
  and noting that $\mathcal{M}_k(\psi)$ is a maximizer, since Assumption 1 guarantees that all elements in  $\mathcal{M}_k(\psi)$ have non-negative expected marginal gains, that is, $\D{e}{\psi} \geq 0,\ \forall e \in \mathcal{M}_k(\psi)$.
\end{proof}

}

\end{document}
