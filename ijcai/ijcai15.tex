%%%% ijcai15.tex

\typeout{Non-monotone Adaptive Submodular Maximization}

% These are the instructions for authors for IJCAI-15.
% They are the same as the ones for IJCAI-11 with superficical wording
%   changes only.

\documentclass{article}
% The file ijcai15.sty is the style file for IJCAI-15 (same as ijcai07.sty).
\usepackage{ijcai15}

% Use the postscript times font!
\usepackage{times}

% the following package is optional:
%\usepackage{latexsym} 

% Following comment is from ijcai97-submit.tex:
% The preparation of these files was supported by Schlumberger Palo Alto
% Research, AT\&T Bell Laboratories, and Morgan Kaufmann Publishers.
% Shirley Jowell, of Morgan Kaufmann Publishers, and Peter F.
% Patel-Schneider, of AT\&T Bell Laboratories collaborated on their
% preparation.

% These instructions can be modified and used in other conferences as long
% as credit to the authors and supporting agencies is retained, this notice
% is not changed, and further modification or reuse is not restricted.
% Neither Shirley Jowell nor Peter F. Patel-Schneider can be listed as
% contacts for providing assistance without their prior permission.

% To use for other conferences, change references to files and the
% conference appropriate and use other authors, contacts, publishers, and
% organizations.
% Also change the deadline and address for returning papers and the length and
% page charge instructions.
% Put where the files are available in the appropriate places.

%-------------------------------------------------------------------------------
% Custom definitions
%-------------------------------------------------------------------------------
\usepackage{etoolbox}
% Define output mode (default: six-page IJCAI)
\newtoggle{short}
\toggletrue{short}
% Uncomment for long version
\togglefalse{short}

\usepackage[usenames,dvipsnames]{xcolor}
\usepackage{amsmath,amssymb,amsthm}
\usepackage{mathtools}                    % For \vcentcolon
\usepackage{xspace}            % Controls space after user-defined command
\usepackage{xparse}            % Optional arguments in commands
\usepackage{enumitem}
\usepackage{fixltx2e}          % For subscripts in normal text
\usepackage{bm}
\usepackage{algorithm}
\usepackage{algorithmic}
\usepackage{pgfplots}
\iftoggle{short}
{
\usepackage{nohyperref}
\usepackage{url}
}
{
\usepackage{flushend}
\usepackage[debug]{hyperref}
\definecolor{mydarkblue}{rgb}{0,0.08,0.45}
\hypersetup{%
  pdftitle={Active Learning for Level Set Estimation},
  pdfauthor={},
  pdfsubject={},
  pdfkeywords={},
  pdfborder=0 0 0,
  pdfpagemode=UseNone,
  colorlinks=true,
  linkcolor=mydarkblue,
  citecolor=mydarkblue,
  filecolor=mydarkblue,
  urlcolor=mydarkblue,
  pdfview=FitH}
}

\newcommand{\todo}[1]{\noindent\texttt{\small\color[rgb]{0.5,0.1,0.1} TODO: #1}}

% Refs
\newcommand{\sectref}[1]{\hyperref[#1]{Section \ref*{#1}}}
\newcommand{\chapref}[1]{\hyperref[#1]{Chapter \ref*{#1}}}
\newcommand{\figref}[1]{\hyperref[#1]{Figure \ref*{#1}}}
\newcommand{\figsref}[1]{\hyperref[#1]{Figures \ref*{#1}}}
\newcommand{\tabref}[1]{\hyperref[#1]{ Table \ref*{#1}}}
\newcommand{\algoref}[1]{\hyperref[#1]{Algorithm \ref*{#1}}}
\newcommand{\theoremref}[1]{\hyperref[#1]{Theorem \ref*{#1}}}
\newcommand{\lemmaref}[1]{\hyperref[#1]{Lemma \ref*{#1}}}
\newcommand{\lemmasref}[1]{\hyperref[#1]{Lemmas \ref*{#1}}}
\newcommand{\corref}[1]{\hyperref[#1]{Corollary \ref*{#1}}}
\newcommand{\asref}[1]{\hyperref[#1]{Assumption \ref*{#1}}}
\newcommand{\eqtref}[1]{\hyperref[#1]{\mbox{(\ref*{#1})}}}
\newcommand{\appref}[1]{\hyperref[#1]{Appendix \ref*{#1}}}
\newcommand{\lineref}[1]{\hyperref[#1]{line \ref*{#1}}}
\newcommand{\linesref}[2]{\hyperref[#1]{lines \ref*{#1}--\ref*{#2}}}
\newcommand{\linsref}[1]{\hyperref[#1]{lines \ref*{#1}}}

% Theory environments
\newtheorem{definition}{Definition}
\newtheorem{theorem}{Theorem}
\newtheorem{lemma}{Lemma}
\newtheorem{cor}{Corollary}
\newtheorem{assumption}{Assumption}

% Algorithm-related
\newcommand{\theHalgorithm}{\arabic{algorithm}}
\renewcommand{\algorithmicrequire}{\textbf{Input:}}
\renewcommand{\algorithmicensure}{\textbf{Output:}}
\newcommand*\LNot{\textbf{not}\xspace}
\newcommand*\LAnd{\textbf{and}\xspace}
\newcommand*\LET[2]{\STATE #1 $\gets$ #2}
\newcommand*\Fcall[1]{\textsc{#1}}
\makeatletter
\newcommand{\setalglineno}[1]{%
  \setcounter{ALC@line}{\numexpr#1-1}}
\makeatother
\renewcommand{\algorithmiccomment}[1]{// #1}
\newcommand{\LINEIF}[2]{%
    \STATE\algorithmicif\ {#1}\ \algorithmicthen\ {#2}%
}
\newcommand{\LINEELSE}[1]{%
    \STATE\algorithmicelse\ {#1}%
}

% Math definitions
\newcommand{\argmax}{\operatornamewithlimits{argmax}}
\newcommand{\argmin}{\operatornamewithlimits{argmin}}
\def\*#1{\bm{#1}}

\newcommand{\twopartdef}[4]
{
	\left\{
		\begin{array}{ll}
			#1\,,& \mbox{if } #2 \\
			#3\,,& \mbox{if } #4
		\end{array}
	\right.
}

\newcommand{\twopartdefo}[3]
{
	\left\{
		\begin{array}{ll}
			#1\,,& \mbox{if } #2 \\
			#3\,,& \mbox{otherwise}
		\end{array}
	\right.
}

\newcommand{\favg}{f_{\mathrm{avg}}}
\newcommand{\dom}{\mathcal{D}}
\newcommand{\defeq}{\vcentcolon=}

% Fix spacing problem with \left and \right
\let\originalleft\left
\let\originalright\right
\renewcommand{\left}{\mathopen{}\mathclose\bgroup\originalleft}
\renewcommand{\right}{\aftergroup\egroup\originalright}

% Math definitions
\newcommand{\smid}{\ \middle\vert\ }
\newcommand{\mmid}{\,\vert\,}

\DeclareDocumentCommand \E { o m o }{%
  \IfValueTF {#1} {%
    \mathbb{E}_{#1}
  }{%
    \mathbb{E}
  }%
  \left[#2
  \IfValueT {#3} {%
    \smid #3
  }%
  \right]
}

\DeclareDocumentCommand \P { o m o }{%
  \IfValueTF {#1} {%
    \mathbb{P}_{#1}
  }{%
    \mathbb{P}
  }%
  \left[#2
  \IfValueT {#3} {%
    \smid #3
  }%
  \right]
}

\newcommand{\D}[2]{\Delta(#1\mmid#2)}
\newcommand{\sdef}[2]{\left\{#1\smid#2\right\}}

\newcommand{\pio}{\pi_{[0]}}
\newcommand{\pii}{\pi_{[i]}}
\newcommand{\pik}{\pi_{[k]}}
\newcommand{\pis}{\pi^*}
\newcommand{\pisi}{\pi^*_{[i]}}
\newcommand{\pisk}{\pi^*_{[k]}}
\newcommand{\pigr}{\pi^{\textrm{g}}}
\newcommand{\pig}{\pi^{\textrm{rg}}}
\newcommand{\pigo}{\pi^{\textrm{rg}}_{[0]}}
\newcommand{\pigi}{\pi^{\textrm{rg}}_{[i]}}
\newcommand{\pigl}{\pi^{\textrm{rg}}_{[\ell]}}
\newcommand{\pigii}{\pi^{\textrm{rg}}_{[i+1]}}
\newcommand{\pigk}{\pi^{\textrm{rg}}_{[k]}}

\newcommand{\uitem}[1]{\item[#1]}

% For citations like: Author et. al [2010]
\newcommand{\citet}[1]{\citeauthor{#1}~\shortcite{#1}}

% Paragraph
\renewcommand{\paragraph}[1]{\vspace{0.3em}\noindent\textbf{#1.}\makebox[0.5em]{}}
%-------------------------------------------------------------------------------

\title{Non-monotone Adaptive Submodular Maximization}
\author{Alkis Gotovos\\
ETH Zurich
\And
Amin Karbasi\\
Yale University
\And
Andreas Krause\\
ETH Zurich}

\begin{document}

\maketitle

\begin{abstract}
\todo{abstract}
\end{abstract}

\section{Introduction}

\subsection{Related work}

\section{Problem Statement and Background}
Assume we are given a finite ground set $E$ and a set $O$ of observable states.
Each item $e \in E$ is associated with a state $o \in O$ through a function $\phi : E \to O$, which is called a realization of the ground set.
In our setting, we assume that the realization $\Phi$ is a random variable with known distribution $p(\phi)$.
Furthermore, we are given an objective function $f : 2^E \times O^E \to \mathbb{R}_{\geq 0}$.
For a set $A \subseteq E$ and a realization $\phi$, the quantity $f(A, \phi)$ expresses the utility we get by selecting subset $A$, if the true realization is $\phi$.

Our goal is to come up with a sequential policy that builds up a set $A \subseteq E$, such that our utility $f(A, \Phi)$ is maximized.
That is, we iteratively select an item $e \in E$ to add to $A$ and observe its state $\Phi(e)$.
In this setting, there are two factors that complicate matters compared to its non-adaptive counterpart.
First, since our utility depends on the (random) realization, we need to maximize the expected utility under the distribution of realizations $p(\phi)$.
Second, the chosen set $A$ itself is a random variable that depends on the realization, since the choices of our policy will change according to each observation $\Phi(e)$, which is, of course, the whole point of adaptivity.
In addition, the policy itself might make random decisions, which is an additional source of randomness for $A$.

To address the above complications, we define a partial realization as a set $ \psi \subseteq E \times O$, which represents the item-observation pairs over a subset of $E$.
In particular, we call this subset the \emph{domain} of $\psi$, which is formally defined as $\dom(\psi) \defeq \sdef{e \in E}{\exists o \in O : (e, o) \in \psi}$.
Additionally, we write $\psi(e) = o$, if $(e, o) \in \psi$, and call $\psi$ \emph{consistent} with realization $\phi$ (denoted by $\phi \sim \psi$), if $\psi(e) = \phi(e)$, for all $e \in \dom(\psi)$, which means that the observations of a subset according to $\psi$ agree with the assignments over the whole ground set according to $\phi$.

Now, we can define a \emph{policy} $\pi$ as a function from partial realizations to a distribution over which item to pick next, formally, $\pi : 2^{E \times O} \to \mathcal{P}(E)$.
The policy terminates when the current partial realization is not in its domain denoted by $\dom(\pi) \subseteq 2^{E \times O}$.
We use the notation $\pi(e\mmid\psi)$ for the probability of picking item $e$ given partial realization $\psi$.
We call $E(\pi, \Phi) \subseteq E$ the set of items that have been selected upon termination of policy $\pi$ under realization $\Phi$.
Note that this set a random variable that depends on both the randomness of the policy, as well as the randomness of the realization.

Finally, we can formalize the performance of a policy $\pi$ by its expected utility $\favg(\pi) \defeq \E[\Phi,\Pi]{f(E(\pi, \Phi), \Phi)}$.
Then, our goal is to come up with a policy that maximizes the expected utility, subject to a cardinality constraint on the number of items to be picked, $|E(\pi, \Phi)| \leq k$.

\subsection{Monotonicity and Submodularity}
\paragraph{Non-adaptive}
Even in the non-adaptive setting, where the realization is fixed and known in advance, or equivalently, the objective function $f : 2^E \to \mathbb{R}_{\geq 0}$ depends only on the chosen subset, the problem of maximizing $f(A)$, subject to a cardinality constraint $|A| \leq k$, is NP-hard in general.
In this setting, the marginal gain of an element $e \in E$ given set $B \subseteq E$ is defined as $f(B \cup \{e\}) - f(B)$.
Intuitively, the marginal gain quantifies the increase in utility if we add $e$ to our selection, given that we have already picked the elements in $B$.
Function $f$ is called monotone if for any $B \subseteq C \subseteq E$, it holds that $f(B) \leq f(C)$, which is equivalent to saying that the marginal gain is always positive.
Furthermore, $f$ is called submodular, if for any $B \subseteq C \subseteq E$ and any $e \in E \setminus C$, it holds that $f(C \cup \{e\}) - f(C) \leq  f(B \cup \{e\}) - f(B)$.
This means that the marginal gain of any element decreases as the given set increases ($C \supseteq B$); in other words, it expresses a behavior of ``diminishing returns''.

In their famous result, \citet{nemhauser78} showed that, if $f$ is normalized ($f(\varnothing) = 0$), monotone, and submodular, then constructing a subset $A$ by picking elements greedily according to their marginal gains, guarantees that $f(A)$ is a ($1 - 1/e$)-approximation to the optimal value.

\paragraph{Adaptive}
In the significantly more complex adaptive setting, the problem of computing an optimal policy is hard to approximate even for seemingly simple classes of objective functions (e.g. linear) \cite{golovin11}.
However, \citet{golovin11} showed that the notions of monotonicity and submodularity can be naturally generalized to this setting, and that, for objective functions that satisfy these properties, a simple adaptive greedy policy is a ($1 - 1/e$)-approximation to the optimal policy in terms of expected utility.

More concretely, the (expected) marginal gain of element $e \in E$ given partial realization $\psi$ can be defined as
\begin{align*}
  \D{e}{\psi} \defeq \E[\Phi]{f\big(\mathcal{D}(\psi) \cup \{e\}, \Phi\big) - f\big(\dom(\psi), \Phi\big)}[\Phi \sim \psi].
\end{align*}
Note that this is a conditional expectation, which only considers realizations that are consistent with $\psi$.
Then, the following definitions are analogous to their non-adaptive counterparts:
\begin{itemize}
\item $f$ is called adaptive monotone, if $\D{e}{\psi} \geq 0$, for all $e \in E$ and all $\psi$ of positive probability,
\item $f$ is called adaptive submodular, if $\D{e}{\psi'} \leq \D{e}{\psi}$, for all $e \in E \setminus \dom(\psi')$ and all $\psi' \supseteq \psi$.
\end{itemize}
The random greedy policy $\pigr$ selects at each step the element $e \in E$ that achieves the highest marginal gain $\D{e}{\psi}$, given all previously selected elements and their corresponding observed states encoded in partial realization $\psi$.

\subsection{Example: Adaptive Graph Cut}

\section{The Adaptive Random Greedy Algorithm}
\todo{Need to mention $2k$ dummy elements added to $E$.}

\begin{algorithm}[!b]
  \caption{Adaptive random greedy}
  \label{alg:rg}
\normalsize{
\begin{algorithmic}[1]
  \REQUIRE ground set $E$, function $f$, distribution $p(\phi)$, cardinality constraint $k$
%  \ENSURE set $S_k \subseteq E$
  \LET{$A$}{$\varnothing$}
  \LET{$\psi$}{$\varnothing$}
  \FOR{$i = 1$ \TO $k$}
    \STATE Compute $\D{e}{\psi}$, for all $e \in E$ \label{lin:marg}
    \LET{$\mathcal{M}_k(\psi)$}{$\displaystyle\argmax_{S \subseteq E, |S| \leq k}\left\{\sum_{e \in S} \D{e}{\psi} \right\}$} \label{lin:argmax}
    \STATE Sample $m$ uniformly at random from $\mathcal{M}_k(\psi)$
    \LET{$A$}{$A \cup \{m\}$}
    \STATE Observe $\Phi(m)$
    \LET{$\psi$}{$\psi \cup \left\{\big(m, \Phi(m)\big)\right\}$}
  \ENDFOR
  \STATE Return $A$
\end{algorithmic}
}
\end{algorithm}

\section{Theoretical Analysis}

\section{Experiments}

\section{Conclusion}

% The file named.bst is a bibliography style file for BibTeX 0.99c
\bibliographystyle{named}
\bibliography{ijcai15}

\iftoggle{short}
{}
{
\clearpage
\onecolumn
\appendix
%\chapter{Proofs}

\section{Preliminaries}

\subsection{Definitions}
\begin{description}[labelindent=0pt,leftmargin=7pt,itemindent=-2pt,itemsep=0pt]
  \uitem{Ground set} $E$
  \uitem{Observation set} $O$
  \uitem{Realization} $\phi : E \to O$
  \uitem{Partial realization} $\psi \subseteq E \times O$ with $\psi(e) \defeq o,\ \forall (e, o) \in \psi$
  \uitem{Partial realiz. domain} $\dom(\psi) \defeq \sdef{e \in E}{\exists o \in O : (e, o) \in \psi}$
  \uitem{Consistency} $\phi \sim \psi \iff \phi(e) = \psi(e),\ \forall e \in \dom(\psi)$
  \uitem{Subrealization} $\psi_1 \subseteq \psi_2$
  \uitem{Probability simplex} $\mathcal{P}(E) \defeq \sdef{\*x \in \mathbb{R}^E}{\sum_{e \in E} x_e = 1,\ x_e \geq 0\ \forall e \in E}$
  \uitem{Policy} $\pi \subseteq 2^{E \times O} \times \mathcal{P}(E)$ with $\pi(e\mid\psi) \defeq p(e),\ \forall (\psi, p) \in \pi$
  \uitem{Policy domain} $\dom(\pi) \defeq \sdef{\psi \in 2^{E \times O}}{\exists p \in \mathcal{P}(E) : (\psi, p) \in \pi}$
  \uitem{Truncated policy} $\pik \subseteq 2^{E \times O} \times \mathcal{P}(E)$ such that $\dom(\pik) = \sdef{\psi \in \dom(\pi)}{|\psi| < k}$ and $\pik(\psi) = \pi(\psi),\ \forall \psi \in \dom(\pik)$
  \uitem{Selected items} $E(\pi, \phi) \subseteq E$
  \uitem{Function} $f : 2^E \times O^E \to \mathbb{R}_{\geq 0}$
  \uitem{Exp. value of policy} $\favg(\pi) \defeq \E[\Phi,\Pi]{f(E(\pi, \Phi), \Phi)}$
  \uitem{Expected gain of element} $\D{e}{\psi} \defeq \E[\Phi]{f(\dom(\psi) \cup \{e\}, \Phi) - f(\dom(\psi), \Phi)}[\Phi \sim \psi]$
  \uitem{Expected gain of policy} $\D{\pi}{\psi} \defeq \E[\Phi,\Pi]{f(\dom(\psi) \cup E(\pi, \Phi), \Phi) - f(\dom(\psi), \Phi)}[\Phi \sim \psi]$
  \uitem{Policy concatenation} $\pi_1 @ \pi_2$ (note that $\favg(\pi_1 @ \pi_2) = \favg(\pi_2 @ \pi_1)$)
  \uitem{Random greedy policy} $\pig$
  \uitem{Random greedy set} $\mathcal{M}_k(\psi) \in \displaystyle\argmax_{S \subseteq E, |S| \leq k}\left\{\sum_{e \in S} \D{e}{\psi} \right\}$
\end{description}

\todo{Add random greedy policy pseudocode.}

\subsection{Problem statement}
We consider the problem of adaptive maximization, i.e., of finding a policy $\pi$ that solves the following problem
\begin{align*}
  \textrm{maximize}&\quad  \favg(\pi)\\
  \textrm{subject to}&\quad  |E(\pi, \phi)| \leq k,\ \forall \phi \in O^E.
\end{align*}

\section{Adaptive monotone case}

\begin{lemma}\label{lem:mon_subm}
  If $f$ is adaptive submodular, then for any policy $\pi$ and any partial realization $\psi$ it holds that
  \begin{align*}
    \D{\pik}{\psi} \leq \sum_{e \in \mathcal{M}_k(\psi)} \D{e}{\psi}.
  \end{align*}
\end{lemma}
\begin{proof}
  The proof is similar to that of Lemma A.9 by~\citet{golovin11}, but presented here in more detail.
  First note that by definition of $\pik$ it holds that $|E(\pik, \phi)| \leq k,$ for all $\phi$, which also implies that $\E[\Phi,\Pi]{|E(\pik, \Phi)|} \leq k$.
  Let $p(e\mmid\psi)$ be the probability that element $e$ will be selected by the truncated policy $\pik$, which is run after having observed partial realization $\psi$, that is,
  \begin{align*}
    p(e\mmid\psi) \defeq \P[\Phi,\Pi]{e \in E(\pik, \Phi)}[\Phi \sim \psi].
  \end{align*}
  It follows that
  \begin{align}
    \label{eq:pre}
    \begin{split}
      k &\geq \E[\Phi, \Pi]{|E(\pik, \Phi)|}\\
        &= \sum_{e \in E}\P[\Phi,\Pi]{e \in E(\pik, \Phi)}\\
        &= \sum_{e \in E}p(e\mmid\psi).
    \end{split}
  \end{align}
  Now, consider the following fractional knapsack problem:
  \begin{align*}
    \textrm{maximize} & \quad g(\*w) \defeq \sum_{e \in E}\D{e}{\psi}w_e\\
    \textrm{subject to} & \quad \sum_{e \in E}w_e \leq k\\
               & \quad 0 \leq w_e \leq 1,\ \forall e \in E.
  \end{align*}
  Note that by \eqref{eq:pre} and the fact that $p(e\mmid\psi)$ are probabilities, it follows that $\*p \defeq \left(p(e\mmid\psi),\ e \in E\right)$ is a feasible vector for the above problem.
  Furthermore, the vector $\*m$ defined as
  \begin{align*}
    m_e = \twopartdefo{1}{e \in \mathcal{M}_k(\psi)}{0}
  \end{align*}
  is an optimal solution. \todo{Could add a proof by contradiction here if this isn't obvious.}
  Therefore, we have
  \begin{align}\label{eq:knap}
                    &\ g(\*p) \leq g(\*m) \notag\\
    \Leftrightarrow &\ \sum_{e \in E}\D{e}{\psi}w_e \leq \sum_{e \in \mathcal{M}_k(\psi)} \D{e}{\psi}
  \end{align}
  
  Let $p(\psi'\mmid\psi)$ be the probability that partial realization $\psi' \supseteq \psi$ will come up when running policy $\pik$ given partial realization $\psi$, that is,
  \begin{align*}
    p(\psi'\mmid\psi) \defeq \P[\Phi, \Pi]{\sdef{\left(e, \Phi(e)\right)}{e \in E(\pik, \Phi)} = \psi'\setminus\psi}[\Phi \sim \psi].
  \end{align*}
  Then, the gain $\D{\pik}{\psi}$ can be bounded as follows:
  \begin{align*}
    \D{\pik}{\psi} &= \sum_{\psi' \in \dom(\pik)}p(\psi'\mmid\psi)\sum_{e \in E}\pik(e\mmid\psi')\D{e}{\psi'}\\
    &\leq \sum_{\psi' \in \dom(\pik)}p(\psi'\mmid\psi)\sum_{e \in E}\pik(e\mmid\psi')\D{e}{\psi} \tag*{(by AS)}\\
    &= \sum_{e \in E}\D{e}{\psi} \sum_{\psi' \in \dom(\pik)}\pik(e\mmid\psi')p(\psi'\mmid\psi)\\
    &= \sum_{e \in E}\D{e}{\psi} p(e\mmid\psi)\\
    &\leq \sum_{e \in \mathcal{M}_k(\psi)} \D{e}{\psi} \tag*{(by \eqref{eq:knap})}.
  \end{align*}
\end{proof}

\begin{lemma}\label{lem:mon_main}
  For any policy $\pi$ and any non-negative integer $i < k$, if $f$ is adaptive submodular, the expected marginal gain obtained at the $i$-th step of random greedy policy $\pig$ can bounded as
  \begin{align*}
    \favg(\pigii) - \favg(\pigi) \geq \frac{1}{k}\left(\favg(\pigi @ \pi) - \favg(\pigi)\right).
  \end{align*}
\end{lemma}
\begin{proof}
  Fix $i < k$ and let $\Psi$ be a random variable denoting the partial realization that results from running the random greedy policy for $i$ steps, distributed as
  \begin{align*}
    \P[\Psi]{\Psi = \psi} = \P[\Phi, \Pi]{\sdef{\left(e, \Phi(e)\right)}{e \in E(\pigi, \Phi)} = \psi}.
  \end{align*}
  Also, let $U_i$ be a random variable denoting the element chosen at the $i$-th step of the random greedy policy. Due to the way the random greedy policy selects the next element at each step, the distribution of $U_{i+1}$ conditioned on some partial realization $\psi$ up to step $i$ is
  \begin{align}\label{eq:ui}
    \P[\Pi]{U_{i+1} = e} = \twopartdefo{1/k}{e \in \mathcal{M}_k(\psi)}{0}.
  \end{align}
  
  Then, for the expected marginal gain at the $i$-th step we have
  \begin{align*}
     &\ \favg(\pigii) - \favg(\pigi)\\
    =&\ \E[\Phi, \Pi]{f(E(\pigii, \Phi), \Phi) - f(E(\pigi, \Phi), \Phi)}\\
    =&\ \E[\Psi, \Phi, \Pi]{f(\dom(\Psi) \cup \{U_{i+1}\}, \Phi) - f(\dom(\Psi), \Phi)}[\Phi \sim \Psi]\\
    =&\ \E[\Psi, \Phi]{\sum_{e \in \mathcal{M}_k(\Psi)}\frac{1}{k}\left[f(\dom(\Psi) \cup \{e\}, \Phi) - f(\dom(\Psi), \Phi)\right]}[\Phi \sim \Psi] \tag*{(by \eqref{eq:ui})}\\
    =&\ \frac{1}{k}\E[\Psi]{\sum_{e \in \mathcal{M}_k(\Psi)}\D{e}{\Psi}}\\
    \geq&\ \frac{1}{k}\E[\Psi]{\D{\pi}{\Psi}} \tag*{(by~\lemmaref{lem:mon_subm})}\\
    =&\ \frac{1}{k}\E[\Psi, \Phi]{f(\dom(\Psi) \cup E(\pi, \Phi), \Phi) - f(\dom(\Psi), \Phi)}[\Phi \sim \Psi]\\
    =& \ \frac{1}{k}\left(\favg(\pigi @ \pi) - \favg(\pigi)\right).
  \end{align*}
\end{proof}

\begin{lemma}\label{lem:mon_mon}
  Function $f$ is adaptive monotone if and only if for all policies $\pi_1$ and $\pi_2$ it holds that
  \begin{align*}
    \favg(\pi_2) \leq \favg(\pi_1 @ \pi_2).
  \end{align*}
\end{lemma}
\begin{proof}
See Lemma A.8 of \citet{golovin11}.
\end{proof}

\begin{theorem}
  If $f$ is adaptive monotone submodular, then for any policy $\pi$ and all integers $i, k \geq 0$ it holds that
  \begin{align*}
    \favg(\pigi) \geq \left(1 - e^{-i/k}\right)\favg(\pik).
  \end{align*}
\end{theorem}
\begin{proof}
  By combining \lemmasref{lem:mon_main} and~\ref{lem:mon_mon} it immediately follows that for all $i$, $k \geq 0$
  \begin{align*}
                   &\ \favg(\pigii) - \favg(\pigi) \geq \frac{1}{k}\left(\favg(\pik) - \favg(\pigi)\right)\\
    \Leftrightarrow&\ \favg(\pigii) \geq \frac{1}{k}\favg(\pik) + \left(1 - \frac{1}{k}\right) \favg(\pigi)\\
    \Leftrightarrow&\ \favg(\pik) - \favg(\pigii) \leq \left(1 - \frac{1}{k}\right)\left(\favg(\pik) - \favg(\pigi)\right)\\
    \Leftrightarrow&\ \favg(\pik) - \favg(\pigi) \leq \left(1 - \frac{1}{k}\right)^{i}\left(\favg(\pik) - \favg(\pigo)\right)\\
    \Leftrightarrow&\ \favg(\pigi) \geq \left(1 - \left(1 - \frac{1}{k}\right)^{i}\right)\favg(\pik)\tag*{(by non-negativity of $f$)}\\
    \Leftrightarrow&\ \favg(\pigi) \geq \left(1 - e^{-i/k}\right)\favg(\pik)\tag*{($1 - x \leq e^{-x},\ \forall x \geq 0$)}
  \end{align*}
\end{proof}

\begin{cor}
  If $f$ is adaptive monotone submodular, then for any policy $\pi$ and any integer $k \geq 0$ it holds that
  \begin{align*}
    \favg(\pigk) \geq (1 - e^{-1})\favg(\pik).
  \end{align*}
\end{cor}

\section{Non-monotone case}
\begin{lemma}\label{lem:buch}
  If $f : 2^E \to \mathbb{R}_{\geq 0}$ is submodular and $A$ is a random subset of $E$, such that each element $e \in E$ is contained in $A$ with probability at most $p$, that is, $\P[A]{e \in A} \leq p,\ \forall e \in E$, then it holds that
  \begin{align*}
    \E[A]{f(A)} \geq (1-p)f(\varnothing).
  \end{align*}
\end{lemma}
\begin{proof}
  See Lemma 2.2 of \citet{buchbinder14}.
\end{proof}

\begin{lemma}
  If $f(\cdot\,, \phi) : 2^E \to \mathbb{R}_{\geq 0}$ is submodular for all $\phi \in O^E$, then for any policy $\pi$ such that each element of $e \in E$ is selected by it with probability at most $p$, that is, $\P[\Pi]{e \in E(\pi, \phi)} \leq p,\ \forall \phi \in O^E,\ \forall e \in E$, the expected value of running $\pi$ can be bounded as follows:
\begin{align*}
  \favg(\pi) \geq (1-p)\,\favg(\pio).
\end{align*}
\end{lemma}
\begin{proof}
  \begin{align*}
    \favg(\pi) &= \E[\Phi,\Pi]{f(E(\pi, \Phi), \Phi)}\\
               &= \E[\Phi]{\E[\Pi]{f(E(\pi, \Phi), \Phi)}}\\
               &\geq \E[\Phi]{(1-p)f(\varnothing, \Phi)} \tag*{(by \lemmaref{lem:buch})}\\
               &= (1-p)\favg(\pio)
  \end{align*}
\end{proof}

\begin{cor}
  If $f(\cdot\,, \phi) : 2^E \to \mathbb{R}_{\geq 0}$ is submodular for all $\phi \in O^E$, then for any policy $\pi$ such that each element of $e \in E$ is selected by it with probability at most $p$, that is, $\P[\Pi]{e \in E(\pi, \phi)} \leq p,\ \forall \phi \in O^E,\ \forall e \in E$, and any policy $\pi'$, the expected value of running $\pi'@\pi$ can be bounded as follows:
\begin{align*}
  \favg(\pi' @ \pi) \geq (1-p)\favg(\pi').
\end{align*}
\end{cor}
\todo{Add proof?}

\begin{lemma}
  After running the random greedy policy for $i$ steps, the probability of any element $e \in E$ having been selected can be bounded for every $\phi \in O^E$ as follows:
\begin{align*}
  \P[\Pi]{e \in E(\pigi, \phi)} \leq \left(1 - \frac{1}{k}\right)^i
\end{align*}
\end{lemma}
\begin{proof}
\end{proof}

\begin{theorem}
  If $f(\cdot\,, \phi)$ is submodular for all $\phi \in O^E$, then for any policy $\pi$ and all integers $i, k \geq 0$ it holds that
  \begin{align*}
    \favg(\pigi) \geq \frac{i}{k}\left(1 - \frac{1}{k}\right)^{i-1}\favg(\pik).
  \end{align*}
\end{theorem}
\begin{proof}
\end{proof}

\begin{cor}
  If $f(\cdot\,, \phi)$ is submodular for all $\phi \in O^E$, then for any policy $\pi$ and any integer $k \geq 0$ it holds that
  \begin{align*}
    \favg(\pigk) \geq e^{-1}\favg(\pik).
  \end{align*}
\end{cor}

}

\end{document}
